\documentclass[10pt]{article}
\usepackage[utf8]{inputenc}
\usepackage[T1]{fontenc}
\usepackage{amsmath}
\usepackage{amsfonts}
\usepackage{amssymb}
\usepackage[version=4]{mhchem}
\usepackage{stmaryrd}

\begin{document}
\begin{enumerate}
  \setcounter{enumi}{5}
  \item Let $\left\{a_{\mathrm{n}}\right\}_{\mathrm{n}=0}^{\infty}$ be a sequence such that $a_{0}=a_{1}=0$ and $a_{\mathrm{n}+2}=3 a_{\mathrm{n}+1}-2 a_{\mathrm{n}}+1, \forall \mathrm{n} \geq 0$.
\end{enumerate}

Then $a_{25} a_{23}-2 a_{25} a_{22}-2 a_{23} a_{24}+4 a_{22} a_{24}$ is equal to:

(A) 483

(B) 528

(C) 575

(D) 624

Official Ans. by NTA (B)

Ans. (B)

Sol. $a_{0}=0, a_{1}=0$

$a_{\mathrm{n}+2}=3 a_{\mathrm{n}+1}-2 a_{\mathrm{n}+1}: \mathrm{n} \geq 0$

$a_{\mathrm{n}+2}-a_{\mathrm{n}+1}=2\left(a_{\mathrm{n}+1}-\mathrm{a}_{\mathrm{n}}\right)+1$

$\mathrm{n}=0 \quad a_{2}-a_{1}=2\left(a_{1}-a_{0}\right)+1$

$\mathrm{n}=1 \quad a_{3}-a_{2}=2\left(a_{2}-a_{1}\right)+1$

$\mathrm{n}=2 \quad a_{4}-a_{3}=2\left(a_{3}-a_{2}\right)+1$

$\mathrm{n}=\mathrm{n} \quad a_{\mathrm{n}+2}-a_{\mathrm{n}+1}=2\left(a_{\mathrm{n}+1}-a_{\mathrm{n}}\right)+1$

$\left(a_{\mathrm{n}+2}-a_{1}\right)-2\left(a_{\mathrm{n}+1}-a_{0}\right)-(\mathrm{n}+1)=0$

$a_{\mathrm{n}+2}=2 a_{\mathrm{n}+1}+(\mathrm{n}+1)$

$\mathrm{n} \rightarrow \mathrm{n}-2$

$a_{\mathrm{n}}-2 a_{\mathrm{n}-1}=\mathrm{n}-1$

Now $\mathrm{a}_{25} \mathrm{a}_{23}-2 \mathrm{a}_{25} \mathrm{a}_{22}-2 \mathrm{a}_{23} \mathrm{a}_{24}+4 \mathrm{a}_{22} \mathrm{a}_{24}$

$=\left(a_{25}-2 a_{24}\right)\left(a_{23}-2 a_{22}\right)=(24)(22)=528$

\begin{enumerate}
  \setcounter{enumi}{6}
  \item $\sum_{\mathrm{r}=1}^{20}\left(\mathrm{r}^{2}+1\right)(\mathrm{r} !)$ is equal to:

(A) $22 !-21$ !

(B) $22 !-2(21 !)$

(C) $21 !-2(20 !)$

(D) $21 !-20$ !
\end{enumerate}

Official Ans. by NTA (B)

Ans. (B)

Sol. $\sum_{\mathrm{x}=1}^{20}\left(\mathrm{r}^{2}+1\right) \mathrm{r}$ !

$\sum_{\mathrm{x}=1}^{20}\left((\mathrm{r}+1)^{2}-2 \mathrm{r}\right) \mathrm{r}$ !

$\sum_{\mathrm{x}=1}^{20}((\mathrm{r}+1)(\mathrm{r}+1) !-\mathrm{r} . \mathrm{r} !)-\sum_{\mathrm{r}=1}^{20} \mathrm{r} . \mathrm{r}$ !

$\sum_{\mathrm{x}=1}^{20}((\mathrm{r}+1)(\mathrm{r}+1) !-\mathrm{r} . \mathrm{r} !)-\sum_{\mathrm{r}=1}^{20}((\mathrm{r}+1) !-\mathrm{r} !)$

$=(21.21-1)-(\mid 21-1)$

$=20.21 !=22 !-2.21 !$

\begin{enumerate}
  \setcounter{enumi}{7}
  \item For $I(x)=\int \frac{\sec ^{2} x-2022}{\sin ^{2022} x} d x$, if $I\left(\frac{\pi}{4}\right)=2^{1011}$, then

(A) $3^{1010} \mathrm{I}\left(\frac{\pi}{3}\right)-\mathrm{I}\left(\frac{\pi}{6}\right)=0$

(B) $3^{1010} \mathrm{I}\left(\frac{\pi}{6}\right)-\mathrm{I}\left(\frac{\pi}{3}\right)=0$

(C) $3^{1011} \mathrm{I}\left(\frac{\pi}{3}\right)-\mathrm{I}\left(\frac{\pi}{6}\right)=0$

(D) $3^{1011} \mathrm{I}\left(\frac{\pi}{6}\right)-\mathrm{I}\left(\frac{\pi}{3}\right)=0$
\end{enumerate}

Official Ans. by NTA (A)

Ans. (A)

Sol. $\quad \mathrm{I}(\mathrm{x})=\int \sec ^{2} \mathrm{x} \cdot \sin ^{-2022} \mathrm{x} d x-2022 \int \sin ^{-2022} \mathrm{x} d x$

$=\tan \mathrm{x} \cdot(\sin \mathrm{x})^{-2022}+\int^{\text {I }}(2022) \tan \mathrm{x} \cdot(\sin \mathrm{x})^{-2023} \cos \mathrm{xdx}$

$-2022 \int(\sin \mathrm{x})^{-2022} \mathrm{dx}$

$\mathrm{I}(\mathrm{x})=(\tan \mathrm{x})(\sin \mathrm{x})^{-2022}+\mathrm{C}$

At $\mathrm{X}=\pi / 4,2^{1011}=\left(\frac{1}{\sqrt{2}}\right)^{-2022}+\mathrm{C} \therefore \mathrm{C}=0$

Hence $I(x)=\frac{\tan x}{(\sin x)^{2022}}$

$\mathrm{I}(\pi / 6)=\frac{1}{\sqrt{3}\left(\frac{1}{2}\right)^{2022}}=\frac{2^{2022}}{\sqrt{3}}$

$\mathrm{I}(\pi / 3)=\frac{\sqrt{3}}{\left(\frac{\sqrt{3}}{2}\right)^{2022}}=\frac{2^{2022}}{(\sqrt{3})^{2021}}=\frac{1}{3^{1010}} \mathrm{I}\left(\frac{\pi}{6}\right)$

$3^{1010} \mathrm{I}(\pi / 3)=\mathrm{I}(\pi / 6)$

\begin{enumerate}
  \setcounter{enumi}{8}
  \item If the solution curve of the differential equation $\frac{\mathrm{dy}}{\mathrm{dx}}=\frac{\mathrm{x}+\mathrm{y}-2}{\mathrm{x}-\mathrm{y}}$ passes through the point $(2,1)$ and $(k+1,2), k>0$, then
\end{enumerate}

(A) $2 \tan ^{-1}\left(\frac{1}{\mathrm{k}}\right)=\log _{\mathrm{e}}\left(\mathrm{k}^{2}+1\right)$

(B) $\tan ^{-1}\left(\frac{1}{\mathrm{k}}\right)=\log _{\mathrm{e}}\left(\mathrm{k}^{2}+1\right)$

(C) $2 \tan ^{-1}\left(\frac{1}{\mathrm{k}+1}\right)=\log _{\mathrm{e}}\left(\mathrm{k}^{2}+2 \mathrm{k}+2\right)$

(D) $2 \tan ^{-1}\left(\frac{1}{\mathrm{k}}\right)=\log _{\mathrm{e}}\left(\frac{\mathrm{k}^{2}+1}{\mathrm{k}^{2}}\right)$

Official Ans. by NTA (A)

Ans. (A)

Sol. $\frac{\mathrm{dy}}{\mathrm{dx}}=\frac{\mathrm{x}+\mathrm{y}-2}{\mathrm{x}-\mathrm{y}}=\frac{(\mathrm{x}-1)+(\mathrm{y}-1)}{(\mathrm{x}-1)-(\mathrm{y}-1)}$

$\mathrm{x}-1=\mathrm{X}, \mathrm{y}-1=\mathrm{Y}$

$\frac{\mathrm{d} Y}{\mathrm{dX}}=\frac{\mathrm{X}+\mathrm{Y}}{\mathrm{X}-\mathrm{Y}}$

$\mathrm{Y}=\mathrm{VX} \quad \frac{\mathrm{dY}}{\mathrm{dX}}=\mathrm{V}+\mathrm{X} \frac{\mathrm{dV}}{\mathrm{dX}}$

$\mathrm{V}+\mathrm{X} \frac{\mathrm{dV}}{\mathrm{dX}}=\frac{1+\mathrm{V}}{1-\mathrm{V}} \quad \mathrm{X} \frac{\mathrm{dV}}{\mathrm{dX}}=\frac{\mathrm{V}^{2}+1}{1-\mathrm{V}}$

$\int \frac{1-V}{1+V^{2}} d V=\int \frac{d X}{X}$

$\int \frac{\mathrm{dV}}{1+\mathrm{V}^{2}}-\frac{1}{2} \int \frac{2 \mathrm{VdV}}{1+\mathrm{V}^{2}}=\int \frac{\mathrm{dX}}{\mathrm{X}}$

$\tan ^{-1} \mathrm{~V}-\frac{1}{2} \ln \left(1+\mathrm{V}^{2}\right)=\ln \mathrm{X}+\mathrm{c}$

$\tan ^{-1}\left(\frac{Y}{X}\right)-\frac{1}{2} \ln \left(1+\frac{Y^{2}}{X^{2}}\right)=\ln (X)+c$

$\tan ^{-1}\left(\frac{\mathrm{y}-1}{\mathrm{x}-1}\right)-\frac{1}{2} \ln \left(1+\frac{(\mathrm{y}-1)^{2}}{(\mathrm{x}-1)^{2}}\right)=\ln (\mathrm{x}-1)+\mathrm{c}$

Passes through $(2,1)$

$0-\frac{1}{2} \ln 1=\ln 1+\mathrm{c} \therefore \mathrm{c}=0$

Passes through $(\mathrm{k}+1,2)$

$\therefore \tan ^{-1}\left(\frac{1}{\mathrm{k}}\right)-\frac{1}{2} \operatorname{In}\left(1+\frac{1}{\mathrm{k}^{2}}\right)=\ln \mathrm{k}$

$2 \tan ^{-1}\left(\frac{1}{\mathrm{k}}\right)=\operatorname{In}\left(\frac{1+\mathrm{k}^{2}}{\mathrm{k}^{2}}\right)+2 \ln \mathrm{k}$

$2 \tan ^{-1}\left(\frac{1}{\mathrm{k}}\right)=\operatorname{In}\left(1+\mathrm{k}^{2}\right)$

\begin{enumerate}
  \setcounter{enumi}{9}
  \item Let $y=y$ (x) be the solution curve of the differential equation $\frac{d y}{d x}+\left(\frac{2 x^{2}+11 x+13}{x^{3}+6 x^{2}+11 x+6}\right)$ $y=\frac{(x+3)}{x+1}, x>-1$, which passes through the point $(0,1)$. Then y (1) is equal to:

(A) $\frac{1}{2}$

(B) $\frac{3}{2}$

(C) $\frac{5}{2}$

(D) $\frac{7}{2}$
\end{enumerate}

Official Ans. by NTA (B)

Ans. (B)

Sol. $\frac{d y}{d x}+\left(\frac{2 x^{2}+11 x+13}{x^{3}+6 x^{2}+11 x+6}\right) y=\frac{x+3}{x+1}$

$\int p(x) d x$ I.F. $=\mathrm{e}^{\int \mathrm{p}(\mathrm{x}) \mathrm{dx}}$

$\int p(x) d x=\int \frac{\left(2 x^{2}+11 x+13\right) d x}{(x+1)(x+2)(x+3)}$

Using partial fraction

$\frac{2 x^{2}+11 x+13}{(x+1)(x+2)(x+3)}=\frac{A}{x+1}+\frac{B}{x+2}+\frac{C}{x+3}$

$\mathrm{A}=\frac{4}{2}=2$

$\mathrm{B}=1$

$\mathrm{C}=-1$

$\because \int \mathrm{p}(\mathrm{x}) \mathrm{dx}=\mathrm{A} \ln (\mathrm{x}+1)+\mathrm{B} \ln (\mathrm{x}+2)+\mathrm{c} \ln (\mathrm{x}+3)$

$=\ln \left(\frac{(x+1)^{2}(x+2)}{x+3}\right)$

I.F. $=\mathrm{e}^{\int \mathrm{p}(\mathrm{x}) \mathrm{dx}}=\frac{(\mathrm{x}+1)^{2}(\mathrm{x}+2)}{(\mathrm{x}+3)}$

Solution $y(I F)=\int Q .(I F) d x$

$\mathrm{y}\left(\frac{(\mathrm{x}+1)^{2}(\mathrm{x}+2)}{\mathrm{x}+3}\right)=\int\left(\frac{x+3}{x+1}\right) \frac{(\mathrm{x}+1) /(\mathrm{x}+2)}{(x+3)} \mathrm{dx}$

$\mathrm{y}\left(\frac{(\mathrm{x}+1)^{2}(\mathrm{x}+2)}{\mathrm{x}+3}\right)=\frac{\mathrm{x}^{3}}{3}+\frac{3 \mathrm{x}^{2}}{2}+2 \mathrm{x}+\mathrm{c}$

Passes through $(0,1) \mathrm{C}=\frac{2}{3}$

Now put $\mathrm{x}=1$

$\Rightarrow \mathrm{y}(1)=\frac{3}{2}$

\begin{enumerate}
  \setcounter{enumi}{10}
  \item Let $\mathrm{m}_{1}, \mathrm{~m}_{2}$ be the slopes of two adjacent sides of a square of side a such that $a^{2}+11 a+3\left(m_{2}^{2}+m_{2}^{2}\right)=220$. If one vertex of the square is $(10(\cos \alpha-\sin \alpha), 10(\sin \alpha+\cos \alpha))$, where $\alpha \in\left(0, \frac{\pi}{2}\right)$ and the equation of one diagonal is $(\cos \alpha-\sin \alpha) \mathrm{x}+(\sin \alpha+\cos \alpha) \mathrm{y}=10$, then 72 $\left(\sin ^{4} \alpha+\cos ^{4} \alpha\right)+a^{2}-3 a+13$ is equal to:

(A) 119

(B) 128

(C) 145

(D) 155
\end{enumerate}

Official Ans. by NTA (B)

Ans. (B)


\end{document}